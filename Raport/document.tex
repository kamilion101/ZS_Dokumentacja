
%╔════════════════════════════╗
%║		Szablon wykonał		  ║
%║	mgr inż. Dawid Kotlarski  ║
%║		  10.10.2021		  ║
%╚════════════════════════════╝

\documentclass[12pt,a4paper]{mwart}
\usepackage[utf8]{inputenc}
\usepackage{polski}
\usepackage[T1]{fontenc}
\usepackage{amsmath}
\usepackage{amsfonts}
\usepackage{amssymb}
\usepackage{graphicx}
\usepackage{array}
\usepackage{multirow}
\usepackage{geometry}
\usepackage{tabularray}

\geometry{legalpaper, margin=1.5cm}

\renewcommand{\arraystretch}{1.2}

\begin{document}
	
\begin{center}
	\Huge Raport tygodniowy
\end{center}

\begin{table}[h!]
	\centering
	
	\begin{tblr}
		{ || X[0.1\textwidth,l] | X[0.15\textwidth,c] | X[0.15\textwidth,l] | X[0.15\textwidth,c] | X[0.15\textwidth,l] | X[c] || }
		\hline \hline
		\multicolumn{6}{|c|}{PAŃSTWOWA WYŻSZA SZKOŁA ZAWODOWA W NOWYM SĄCZU}											\\
		\multicolumn{6}{|c|}{Instytut Techniczny, Informatyka}															\\ \hline \hline
		Przedmiot:         & \multicolumn{5}{l|}{Programowanie urządzeń mobilnych -- projekt, mgr inż. Dawid Kotlarski} \\ \hline
		Temat:             &  \multicolumn{5}{l|}{Zdrowy Spacer}                                                                      \\ \hline
		Grupa:             & IS-2(s)P3           & Tydzień:          & 14          & Data:          & 12.01.2022         \\ \hline
		Osoby:             & 
		 \multicolumn{5}{l|}{Kamil Pociecha, Nicolas Świątnik}                                                                      \\ \hline \hline
	\end{tblr}
\end{table}

\section{Wykonane zadania}

\textit{
	\newline
- dodanie na stronie z mapą przycisku, który otwiera Google Maps z możliwością wyznaczenia trasy do miejsca docelowego
\newline
- dodanie możliwości wyboru zdjęcia z galerii telefonu, które następnie zostanie wyśietlone w aplikacji
\newline
- w dokumentacji napisany został rozdział 5 z testowaniem i dokończony rozdział 6 z podręcznikiem użytkownika
\newline
} % Usunąć

\section{Niewykonane zadania}
\textit{
	
	
	
	
}
\section{Napotkane problemy}
\textit{
	
	
	
	
}
\section{Zadania na kolejny tydzień}

\textit{\newline
- dodanie opcji wyboru motywu aplikacji 
} % Usunąć 



\end{document}
