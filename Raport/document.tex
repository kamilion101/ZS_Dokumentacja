
%╔════════════════════════════╗
%║		Szablon wykonał		  ║
%║	mgr inż. Dawid Kotlarski  ║
%║		  10.10.2021		  ║
%╚════════════════════════════╝

\documentclass[12pt,a4paper]{mwart}
\usepackage[utf8]{inputenc}
\usepackage{polski}
\usepackage[T1]{fontenc}
\usepackage{amsmath}
\usepackage{amsfonts}
\usepackage{amssymb}
\usepackage{graphicx}
\usepackage{array}
\usepackage{multirow}
\usepackage{geometry}
\usepackage{tabularray}

\geometry{legalpaper, margin=1.5cm}

\renewcommand{\arraystretch}{1.2}

\begin{document}
	
\begin{center}
	\Huge Raport tygodniowy
\end{center}

\begin{table}[h!]
	\centering
	
	\begin{tblr}
		{ || X[0.1\textwidth,l] | X[0.15\textwidth,c] | X[0.15\textwidth,l] | X[0.15\textwidth,c] | X[0.15\textwidth,l] | X[c] || }
		\hline \hline
		\multicolumn{6}{|c|}{PAŃSTWOWA WYŻSZA SZKOŁA ZAWODOWA W NOWYM SĄCZU}											\\
		\multicolumn{6}{|c|}{Instytut Techniczny, Informatyka}															\\ \hline \hline
		Przedmiot:         & \multicolumn{5}{l|}{Programowanie urządzeń mobilnych -- projekt, mgr inż. Dawid Kotlarski} \\ \hline
		Temat:             &  \multicolumn{5}{l|}{Zdrowy Spacer}                                                                      \\ \hline
		Grupa:             & IS-2(s)P3           & Tydzień:          & 11          & Data:          & 15.12.2021         \\ \hline
		Osoby:             & 
		 \multicolumn{5}{l|}{Kamil Pociecha, Nicolas Świątnik}                                                                      \\ \hline \hline
	\end{tblr}
\end{table}

\section{Wykonane zadania}

\textit{
	\newline
- dodanie działajacej mapy, która pokazuje za pomocą znacznika obecną lokalizację
\newline
- w dokumentacji dodanie fragmentów kodu odpowiadających za inicjalizację oraz działanie mapy wraz z krótkimi opisami
} % Usunąć

\section{Niewykonane zadania}

\textit{\newline
- logowanie za pomocą konta Google} % Usunąć

\section{Napotkane problemy}

\textit{\newline
- aplikacja przerywała swoje działanie po otwarciu mapy - problem został rozwiązany poprzez zmiany w kodzie
\newline
- podczas wdrażania projektu na emulatorze pojawiał się błąd "Niezgodność interfejsu android ABI" - rozwiązanie to odznaczenie i ponowne zaznaczenie wszystkich interfejsów w ustawieniach zaawansowanych Androida oraz usunięcie folderów obj i bin z folderu głównego rozwiązania
} % Usunąć

\section{Zadania na kolejny tydzień}

\textit{\newline
- logowanie za pomocą konta Google 
\newline
- dodanie bazy danych do przechowywania zdjęć przy pomocy pakietu Nuget sqlite-net-pcl
} % Usunąć 



\end{document}
