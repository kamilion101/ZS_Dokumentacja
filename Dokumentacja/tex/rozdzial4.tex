	\newpage
\section{Implementacja}		%4
%Wkleić szkielet kodu, wraz z komentarzami. Opisać zmienne, struktury do czego służą. Opisać procedury, metody co wykonują. Opisać nowe zdefiniowane klasy. Opisać dziedziczenie. Opisać nowo utworzone pliki za co odpowiadają.
\textbf{Tworznie Menu:} \newline
Do stworzenia menu użyty został \textbf{Xamarin.forms Shell}, który zmiejsza złożoność tworzenia aplikacji, oferując podstawowe funkcje. Obejmuje on wspólne środowisko użytkownika nawigacji, schematu nawigacji i zintegrowanej procedury obsługi wyszukiwania.
\newline
\newline
\textbf{Dodawanie strony do menu na przykładzie elementu wyniki:}
\newline
W pliku \textbf{MainPage.xaml}:
\newline
\newline
\textbf{[FlyoutItem Title="MyTabAPP" \newline
\hspace{1cm}Shell.TabBarIsVisible="False" \newline
\hspace{1cm}FlyoutDisplayOptions="AsMultipleItems"] \newline
\hspace{0.5cm}[ShellContent Title="Wyniki" Icon="ic\_wyniki" \newline
ContentTemplate="\{DataTemplate local:Wyniki\}"/] \newline
[/FlyoutItem]} \newline
\newline
, gdzie "ic\_wyniki" to nazwa ikonki a "local:Wyniki" to odnośnik do plików strony "Wyniki", które znajdują się w folderze głównym projektu. 
\newline
\newline
\textbf{Dodawanie ikonek do projektu:} \newline
Ikonki pobrane zostały z \textbf{Android Asset Studio} w formacie ".png". \newline
Aby użyć ikonki w projekcie należy umieścić je w dwóch osobnych miejscach. Dla androida jest to folder \textbf{drawable} znajdujący się w folderze resources a dla systemu iOS folder \textbf{resources}.
\newline \newline
\newpage
Działanie menu bocznego w emulatorze Android 8.1:
\begin{figure}[!htb]
	\begin{center}
		\includegraphics[width=6cm]{rys/ZSmenu.png}
		\caption{Widok menu bocznego}
		\label{rys:rysunek007}
	\end{center}
\end{figure}
\newline \newline
\begin{figure}[!htb]
	\begin{center}
		\includegraphics[width=5cm]{rys/ZSotwartastrona.png}
		\caption{Widok strony otwartej po wybraniu danej opcji z menu}
		\label{rys:rysunek008}
	\end{center}
\end{figure}
 \newpage


