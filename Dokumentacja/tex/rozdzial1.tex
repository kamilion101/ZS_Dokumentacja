	\newpage
\section{Ogólne określenie wymagań}		%1
%Ogólne określenie wymagań i zakresu programu (Czyli zleceniodawca określa wymagania programu)

\hspace{1cm} Aplikacja ma się nazywać Zdrowy Spacer a jej grupą docelową mają być osoby aktywne fizycznie, które chcą kontrolować swoje wyniki.


\subsection{Podstawowe wymagania}  %1.1       

%większe wcięcie
\hspace{1cm} Aplikacja powinna ona spełniać podstawowe funkcjonalności takie jak: pomiary przebytych odległości, liczbę przebytych kroków, liczbę spalonych kalorii.%\footnote{Przykład odnośnika do książki\cite{legierski}.}.

\subsection{Dodatkowe wymagania}  %1.2       

%większe wcięcie
\hspace{1cm} Dodatkowo chcielibyśmy aby aplikacja rzutowała całą trasę na mapy, miała chociażby zapisywaną historię poprzednich tras do wglądu dla użytkownika, oczywiście umożliwiała ustalenie celu czy to podróży lub przebytych kroków/spalonych kalorii. Jako dodatkową funkcję przewidujemy też możliwość robienia zdjęć osiągniętego już celu, czyli np. zrobienie zdjęcia szczytu góry jako potwierdzenie osiągnięcia swojego celu. Chcielibyśmy również umożliwić naszym klientom logowanie się poprzez konta typu Facebook czy Google, co mogło by zautomatyzować proces logowania. Kolejną opcja którą przewidujemy by znalazła się w zamawianej przez nas aplikacji jest motyw typu jasny/ciemny dla uprzyjemnienia samego doświadczenia z korzystania z niej. Oczywiście aplikacja powinna mieć możliwość bezproblemowego działania w tle jak i również ewentualnego wykrycia ruchu bez jej inicjacji co przełoży się na to że sama zacznie rejestrować nasz spacer oraz aktywności.

%rysunek
%	\begin{figure}[!htb]
%	\begin{center}
%		\includegraphics[width=2cm]{rys/logo.png}
%		\caption{Rysunek001}
%		\label{rys:rysunek001}
%	\end{center}
%\end{figure}

%Tutaj może coś być wpisane. \\Tutaj może coś być wpisane\footnote{Przykład odnośnika do strony www\cite{www1}.}. 

%tabelka
%\begin{tabela}
	%uwaga: w nawiasach [] nie może być odnośnika do literatury, jeżeli w dokumencie jest spis rysunków na początku, a spis literatury jest w kolejności cytowania (zmienia to numeracji)
%	{Tablica001}	%opis w spisie tabel
%	{Tablica001.}	%opis przy tabeli
%	{
%		\begin{tabular}{|c|c|} \hline
%			$U_n$ & $I_{zw}$ \\ \hline
%			$kV$  & $\%$      \\ \hline
%			7.2 & 100 \\ \hline
%		\end{tabular}
%	}
%	\label{tab:tablica001}
%\end{tabela}

%Tutaj może coś być wpisane. Tutaj może coś być wpisane. Tutaj może coś być wpisane.